% !TEX root = ../latexnote.tex
\chapter{图片}\label{chap:figure}
\section{简述}\label{sec:figure-intro}
\LaTeX{} 本身不支持插图功能,需要借助graphicx宏包。

使用 latex + dvipdfmx 编译命令时,调用 graphicx 宏包时要指定 dvipdfmx 选项\footnote{早期常使用 latex + dvips 组合命令,后者将 .dvi 文件转为 .ps 文件(PostScript),可进一步通过 ps2pdf 工具生成 PDF。dvips 和 dvipdfmx 在图形、颜色、超链接等功能的实现上有差别,而 \LaTeX{} 无法识别用户是用 dvips 还是dvipdfmx,所以要指定选项(缺省为 dvips)。};而使用 pdflatex 或 xelatex 命令编译时不需要。

下表给出了不同编译命令支持的图片格式:

\begin{table}[htp]
    \centering
    \caption{各种编译方式支持的主流图片格式}\label{tbl:figure-format}
    \begin{tabular}{*{3}{l}}
        \hline
        \textbf{格式}              & \textbf{矢量图} & \textbf{位图}      \\
        \hline
        {latex + dvipdfmx}       & {.eps}       & N/A              \\
        \quad (调用 {bmpsize} 宏包 ) & {.eps .pdf}  & {.jpg .png .bmp} \\[.3\baselineskip]
        {pdflatex}               & {.pdf}       & {.jpg .png}      \\
        \quad (调用 {epstopdf} 宏包) & {.pdf .eps}  & {.jpg .png}      \\[.3\baselineskip]
        {xelatex}                & {.pdf .eps}  & {.jpg .png .bmp} \\
        \hline
    \end{tabular}
    \begin{quote}\footnotesize
        注:在较新的 \TeX{} 发行版中,{latex + dvipdfmx} 和 {pdf\-latex} 命令可不依赖宏包,支持原来需要宏包扩展的图片格式
        (但 {pdf\-latex} 命令仍不支持 {.bmp} 格式的位图)。
    \end{quote}
\end{table}

引入graphicx 宏包后,可使用 \lstinline|\includegraphics| 命令插入图片,其语法为:


\begin{lstlisting}
\includegraphics[options]{file} % options可指定图片属性,如width=
\end{lstlisting}


其中,options 可选参数有:

\begin{table}[htp]
    \centering
    \caption{ \lstinline{\includegraphics} 命令的可选参数}\label{tbl:graphics-options}
    \begin{tabular}{lp{18em}}
        \hline
        \textbf{参数}     & \textbf{含义}          \\
        \hline
        width={width}   & 将图片缩放到宽度为{width}     \\
        height={height} & 将图片缩放到高度为{height}    \\
        scale={scale}   & 将图片相对于原尺寸缩放{scale} 倍 \\
        angle={angle}   & 将图片逆时针旋转{angle} 度    \\
        \hline
    \end{tabular}
\end{table}

\section{浮动体}\label{sec:figure-float}

浮动体将图、表与其标题定义为整体,然后动态排版,以解决图、表卡在换页处造成的过长的垂直空白的问题。图片的浮动体是figure。

下面是一个例子:
\begin{lstlisting}
\begin{figure}[htbp]
    \centering
    \includegraphics[width=0.8\textwidth]{example-image}
    \caption{This is an example image.}
    \label{fig:example-image}
\end{figure}
\end{lstlisting}

参数htbp表示浮动体的位置,h表示插入此处、t表示在页面上端,b表示在页面下端、p表示允许浮动体单开一页。
