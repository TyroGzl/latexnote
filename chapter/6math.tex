% !TEX root = ../latexnote.tex
% Definitions
\def\lsym{$\mathsurround=0pt {}^\ell$}
\def\LSYM    #1{$#1$     & \texttt{\string#1}\lsym}

\def\SYM     #1{$#1$     & \texttt{\string#1}}
\def\BIGSYM  #1{$#1$     & $\displaystyle #1$ & \texttt{\string#1}}
\def\ACC   #1#2{$#1{#2}$ & \texttt{\string#1}\{#2\}}
\def\DEL     #1{$\big#1 \bigg#1$ & \texttt{\string#1}}

\def\AMSSYM  #1{$#1$     & \texttt{\string#1}}
% These symbols rely on `amsmath' package
\def\AMSM    #1{$#1$     & \textcolor{blue}{\texttt{\string#1}}}
\def\AMSACC#1#2{$#1{#2}$ & \textcolor{blue}{\texttt{\string#1}}\{#2\}}
\def\AMSBIG  #1{$#1$     & $\displaystyle #1$ & \textcolor{blue}{\texttt{\string#1}}}

\def\SC      #1{#1       & \texttt{\string#1}}

\newenvironment{symbols}[1]%
{\small\def\arraystretch{1.1}
    \begin{tabular}{@{}#1@{}}}%
        {\end{tabular}}

\DeclareRobustCommand*\amscmd[1]{\textcolor{blue}{\lstinline{#1}}}
\DeclareRobustCommand*\amsenv[1]{\textcolor{blue}{\lstinline{#1}}}

\chapter{数学公式}\label{chap:math}
\section{行内公式}\label{sec:inline}
行内公式使用\lstinline|\$|和\lstinline|\$|包裹,如:

\begin{codeshow}
    行内公式测试$a=b$
\end{codeshow}

\section{行间公式}\label{sec:display}
行间公式使用\lstinline|\begin{equation}|和\lstinline|\end{equation}|包裹,如:

\begin{codeshow}
    \begin{equation}
        a+b=c
    \end{equation}
\end{codeshow}

\lstinline{equation} 环境为公式自动生成一个编号,这个编号可以用 \lstinline{label} 和 \lstinline{ref} 生成交叉引用,
\lstinline{amsmath} 的 \amscmd{eqref} 命令甚至为引用自动加上圆括号;还可以用 \amscmd{tag} 命令手动修改公式的编号,
或者用 \amscmd{notag} 命令取消为公式编号(与之基本等效的命令是 \lstinline{nonumber})。

\begin{codeshow}
    The Pythagorean theorem is:
    \begin{equation}
        a^2 + b^2 = c^2 \label{pythagorean}
    \end{equation}
    Equation \eqref{pythagorean} is
    called `Gougu theorem' in Chinese.

    It's wrong to say
    \begin{equation}
        1 + 1 = 3 \tag{dumb}
    \end{equation}
    or
    \begin{equation}
        1 + 1 = 4 \notag
    \end{equation}
\end{codeshow}


使用\lstinline|\begin{equation*}|和\lstinline|\end{equation*}|包裹的公式不带编号,如:

\begin{codeshow}
    \begin{equation*}
        a+b=c
    \end{equation*}
\end{codeshow}

\section{数学符号}\label{sec:math-symbols}
\subsection{指数、上下标和导数}\label{subsec:math-scripts}
在 \LaTeX{} 中用 \texttt\textasciicircum 和 \texttt\textunderscore 标明上下标。
注意上下标的内容(子公式)一般需要\textbf{用花括号包裹},否则上下标只对后面的一个符号起作用。

\begin{codeshow}
    $p^3_{ij} \qquad
        m_\mathrm{Knuth}\qquad
        \sum_{k=1}^3 k $\\[5pt]
    $a^x+y \neq a^{x+y}\qquad
        e^{x^2} \neq {e^x}^2$
\end{codeshow}

{导数符号\texttt' ($a'$)}
导数符号\texttt'(${}'$)是一类特殊的上标,可连续使用,但只能在\textbf{其后}添加其他上标:

\begin{codeshow}
    $f(x) = x^2 \quad f'(x)
        = 2x \quad f''^{2}(x) = 4$
\end{codeshow}

\subsection{分式和根式}\label{subsec:frac-sqrt}
分式使用 \lstinline|\frac{分子}{分母}| 来书写。分式的大小在行间公式中是正常大小,而在行内被极度压缩。
\lstinline{amsmath} 提供了方便的命令 \amscmd{dfrac} 和 \amscmd{tfrac},令用户能够在行内使用正常大小的分式,或是反过来。

\begin{codeshow}
    In display style:
    \[
        3/8 \qquad \frac{3}{8}
        \qquad \tfrac{3}{8}
    \]
    In text style:
    $1\frac{1}{2}$~hours \qquad
    $1\dfrac{1}{2}$~hours
\end{codeshow}

一般的根式使用 \lstinline|\sqrt{ }|;表示 $n$ 次方根时写成 \lstinline|\sqrt[n]{ }|。

\begin{codeshow}
    $\sqrt{x} \Leftrightarrow x^{1/2}
        \quad \sqrt[3]{2}
        \quad \sqrt{x^{2} + \sqrt{y}}$
\end{codeshow}


特殊的分式形式,如二项式结构,由 \lstinline{amsmath} 宏包的 \amscmd{binom} 命令生成:

\begin{codeshow}
    Pascal's rule is
    \[
        \binom{n}{k} =\binom{n-1}{k}
        + \binom{n-1}{k-1}
    \]
\end{codeshow}

\section{符号表}\label{sec:math-tables}

有几个注意事项:
\begin{enumerate}
    \item 本节选自\href{https://github.com/CTeX-org/lshort-zh-cn/tree/master}{The Not So Short Introduction To LaTeX2ε (Chinese Edition)}
    \item \textcolor{blue}{蓝色}的命令依赖 \lstinline{amsmath} 宏包(非 \lstinline{amssymb} 宏包);
    \item 带有角标\lsym 的符号命令依赖 \lstinline{latexsym} 宏包。
\end{enumerate}

\subsection{\LaTeX{} 普通符号}
\subsubsection{文本/数学模式通用符号}
\begin{table}[htp]
    \centering
    \caption{文本/数学模式通用符号}\label{tbl:general-syms}
    \begin{quote}\footnotesize%
        这些符号可用于文本和数学模式。
    \end{quote}
    \begin{symbols}{*4{cl}}
        \hline
        \SC{\{}    &  \SC{\}}  &  \SC{\$}         &  \SC{\%}               \\
        \SC{\dag}  &  \SC{\S}  &  \SC{\copyright} &  \SC{\dots}            \\
        \SC{\ddag} &  \SC{\P}  &  \SC{\pounds}    &                        \\
        \hline
    \end{symbols}
\end{table}

\subsubsection{希腊字母}
\begin{table}[H]
    \centering
    \caption{希腊字母} \label{tbl:math-greek}
    \begin{quote}\footnotesize%
        % \cmd{Alpha},\cmd{Beta} 等希腊字母符号不存在,因为它们和拉丁字母 A,B 等一模一样;
        % 小写字母里也不存在 \cmd{omicron},直接用拉丁字母 $o$ 代替。
    \end{quote}
    \begin{symbols}{*4{cl}}
        \hline
        \SYM{\alpha}     & \SYM{\theta}     & \SYM{o}          & \SYM{\upsilon}  \\
        \SYM{\beta}      & \SYM{\vartheta}  & \SYM{\pi}        & \SYM{\phi}      \\
        \SYM{\gamma}     & \SYM{\iota}      & \SYM{\varpi}     & \SYM{\varphi}   \\
        \SYM{\delta}     & \SYM{\kappa}     & \SYM{\rho}       & \SYM{\chi}      \\
        \SYM{\epsilon}   & \SYM{\lambda}    & \SYM{\varrho}    & \SYM{\psi}      \\
        \SYM{\varepsilon}& \SYM{\mu}        & \SYM{\sigma}     & \SYM{\omega}    \\
        \SYM{\zeta}      & \SYM{\nu}        & \SYM{\varsigma}  &                 \\
        \SYM{\eta}       & \SYM{\xi}        & \SYM{\tau}       &                 \\[1ex]
        \SYM{\Gamma}     & \SYM{\Lambda}    & \SYM{\Sigma}     & \SYM{\Psi}      \\
        \SYM{\Delta}     & \SYM{\Xi}        & \SYM{\Upsilon}   & \SYM{\Omega}    \\
        \SYM{\Theta}     & \SYM{\Pi}        & \SYM{\Phi}       &                 \\[1ex]
        \AMSM{\varGamma} & \AMSM{\varLambda}& \AMSM{\varSigma}  & \AMSM{\varPsi}      \\
        \AMSM{\varDelta} & \AMSM{\varXi}    & \AMSM{\varUpsilon}& \AMSM{\varOmega}    \\
        \AMSM{\varTheta} & \AMSM{\varPi}    & \AMSM{\varPhi}    &                 \\
        \hline
    \end{symbols}
\end{table}


\subsubsection{二元关系符}
\begin{table}[H]
    \centering
    \caption{二元关系符} \label{tbl:math-rel}
    \begin{quote}\footnotesize%
        所有的二元关系符都可以加 \lstinline{not} 前缀得到相反意义的关系符,例如 \lstinline{not}\texttt{=} 就得到不等号(同 \lstinline{ne})。
    \end{quote}
    \begin{symbols}{*3{cl}}
        \hline
        \SYM{<}              & \SYM{>}                    & \SYM{=}          \\
        \SYM{\leq} or \verb|\le|   & \SYM{\geq} or \verb|\ge| & \SYM{\equiv}     \\
        \SYM{\ll}            & \SYM{\gg}                  & \SYM{\doteq}     \\
        \SYM{\prec}          & \SYM{\succ}                & \SYM{\sim}       \\
        \SYM{\preceq}        & \SYM{\succeq}              & \SYM{\simeq}     \\
        \SYM{\subset}        & \SYM{\supset}              & \SYM{\approx}    \\
        \SYM{\subseteq}      & \SYM{\supseteq}            & \SYM{\cong}      \\
        \LSYM{\sqsubset}     & \LSYM{\sqsupset}           & \LSYM{\Join}     \\
        \SYM{\sqsubseteq}    & \SYM{\sqsupseteq}          & \SYM{\bowtie}    \\
        \SYM{\in}            & \SYM{\ni}, \verb|\owns|      & \SYM{\propto}    \\
        \SYM{\vdash}         & \SYM{\dashv}               & \SYM{\models}    \\
        \SYM{\mid}           & \SYM{\parallel}            & \SYM{\perp}      \\
        \SYM{\smile}         & \SYM{\frown}               & \SYM{\asymp}     \\
        \SYM{:}              & \SYM{\notin}               & \SYM{\neq} or \verb|\ne| \\
        \hline
    \end{symbols}
\end{table}

\subsubsection{二元运算符}
\begin{table}[H]
    \centering
    \caption{二元运算符}\label{tbl:math-op}
    \begin{symbols}{*3{cl}}
        \hline
        \SYM{+}              & \SYM{-}              &                     \\
        \SYM{\pm}            & \SYM{\mp}            & \SYM{\triangleleft} \\
        \SYM{\cdot}          & \SYM{\div}           & \SYM{\triangleright}\\
        \SYM{\times}         & \SYM{\setminus}      & \SYM{\star}         \\
        \SYM{\cup}           & \SYM{\cap}           & \SYM{\ast}          \\
        \SYM{\sqcup}         & \SYM{\sqcap}         & \SYM{\circ}         \\
        \SYM{\vee}, \verb|\lor| & \SYM{\wedge},\verb|\land|  & \SYM{\bullet}   \\
        \SYM{\oplus}         & \SYM{\ominus}        & \SYM{\diamond}      \\
        \SYM{\odot}          & \SYM{\oslash}        & \SYM{\uplus}        \\
        \SYM{\otimes}        & \SYM{\bigcirc}       & \SYM{\amalg}        \\
        \SYM{\bigtriangleup} &\SYM{\bigtriangledown}& \SYM{\dagger}       \\
        \LSYM{\lhd}          & \LSYM{\rhd}          & \SYM{\ddagger}      \\
        \LSYM{\unlhd}        & \LSYM{\unrhd}        & \SYM{\wr}           \\
        \hline
    \end{symbols}
\end{table}

\subsubsection{巨算符}
\begin{table}[H]
    \centering
    \caption{巨算符}\label{tbl:math-bigop}
    \def\arraystretch{2.5}
    \begin{symbols}{*3{ccl}}
        \hline
        \BIGSYM{\sum}      & \BIGSYM{\bigcup}   & \BIGSYM{\bigvee}  \\
        \BIGSYM{\prod}     & \BIGSYM{\bigcap}   & \BIGSYM{\bigwedge} \\
        \BIGSYM{\coprod}   & \BIGSYM{\bigsqcup} & \BIGSYM{\biguplus} \\
        \BIGSYM{\int}      & \BIGSYM{\oint}     & \BIGSYM{\bigodot} \\
        \BIGSYM{\bigoplus} & \BIGSYM{\bigotimes} & \\
        \AMSBIG{\iint}     & \AMSBIG{\iiint}    & \AMSBIG{\iiiint}  \\
        \AMSBIG{\idotsint} &                    & \\
        \hline
    \end{symbols}
\end{table}

\subsubsection{数学重音符号}
\begin{table}[H]
    \centering
    \caption{数学重音符号}\label{tbl:math-accents}
    \begin{quote}\footnotesize%
        最后一个 \lstinline{\wideparen} 依赖 \lstinline{yhmath} 宏包。
    \end{quote}
    \begin{symbols}{*3{cl}}
        \hline
        \ACC{\hat}{a}   & \ACC{\check}{a} & \ACC{\tilde}{a}       \\
        \ACC{\acute}{a} & \ACC{\grave}{a} & \ACC{\breve}{a}       \\
        \ACC{\bar}{a}   & \ACC{\vec}{a}   & \ACC{\mathring}{a}    \\
        \ACC{\dot}{a}   & \ACC{\ddot}{a}  & \AMSACC{\dddot}{a}    \\
        \AMSACC{\ddddot}{a} \\[1ex]
        \ACC{\widehat}{AAA} & \ACC{\widetilde}{AAA} & \ACC{\wideparen}{AAA} \\
        \hline
    \end{symbols}
\end{table}

\subsubsection{箭头}
\begin{table}[H]
    \centering
    \caption{箭头} \label{tbl:math-arrows}
    \begin{symbols}{*2{cl}}
        \hline
        \SYM{\leftarrow} or \verb|\gets| & \SYM{\longleftarrow} \\
        \SYM{\rightarrow} or \verb|\to|   & \SYM{\longrightarrow} \\
        \SYM{\leftrightarrow}    & \SYM{\longleftrightarrow} \\
        \SYM{\Leftarrow}         & \SYM{\Longleftarrow}     \\
        \SYM{\Rightarrow}        & \SYM{\Longrightarrow}    \\
        \SYM{\Leftrightarrow}    & \SYM{\Longleftrightarrow}\\
        \SYM{\mapsto}            & \SYM{\longmapsto}        \\
        \SYM{\hookleftarrow}     & \SYM{\hookrightarrow}    \\
        \SYM{\leftharpoonup}     & \SYM{\rightharpoonup}    \\
        \SYM{\leftharpoondown}   & \SYM{\rightharpoondown}  \\
        \SYM{\rightleftharpoons} & \SYM{\iff}               \\
        \SYM{\uparrow}           & \SYM{\downarrow} \\
        \SYM{\updownarrow}       & \SYM{\Uparrow} \\
        \SYM{\Downarrow}         & \SYM{\Updownarrow} \\
        \SYM{\nearrow}           & \SYM{\searrow} \\
        \SYM{\swarrow}           & \SYM{\nwarrow} \\
        \LSYM{\leadsto}          &              \\
        \hline
    \end{symbols}
\end{table}

\subsubsection{作为重音的箭头符号}
\begin{table}[H]
    \centering
    \caption{作为重音的箭头符号}  \label{tbl:math-arrow-accents}
    \begin{symbols}{*2{cl}}
        \hline
        \ACC{\overrightarrow}{AB}     & \AMSACC{\underrightarrow}{AB}     \\
        \ACC{\overleftarrow}{AB}      & \AMSACC{\underleftarrow}{AB}      \\
        \AMSACC{\overleftrightarrow}{AB} & \AMSACC{\underleftrightarrow}{AB} \\
        \hline
    \end{symbols}
\end{table}

\subsubsection{定界符}
\begin{table}[H]
    \centering
    \caption{定界符}\label{tbl:math-delims}
    \begin{quote}\footnotesize%
        \lstinline{amsmath} 还定义了 \lstinline{lvert}、\lstinline{rvert} 和 \lstinline{lVert}、\lstinline{rVert},
        分别作为 \lstinline{vert} 和 \lstinline{Vert} 对应的开符号(左侧)和闭符号(右侧)的命令。
    \end{quote}
    \begin{symbols}{*4{cl|}}
        \hline
        \SYM{(}                  & \SYM{)}                  & \SYM{\uparrow}     & \SYM{\downarrow}   \\
        \SYM{[}  or \verb|\lbrack| & \SYM{]}  or \verb|\rbrack| & \SYM{\Uparrow}     & \SYM{\Downarrow}   \\
        \SYM{\{} or \verb|\lbrace| & \SYM{\}} or \verb|\rbrace| & \SYM{\updownarrow} & \SYM{\Updownarrow} \\
        \SYM{|}  or \verb|\vert|   & \SYM{\|} or \verb|\Vert|   & \SYM{\lceil}       & \SYM{\rceil}       \\
        \SYM{\langle}            & \SYM{\rangle}            & \SYM{\lfloor}      & \SYM{\rfloor}      \\
        \SYM{/}                  & \SYM{\backslash} \\
        \hline
    \end{symbols}
\end{table}

\subsubsection{用于行间公式的大定界符}
\begin{table}[H]
    \centering
    \caption{用于行间公式的大定界符}\label{tbl:math-large-delims}
    \def\arraystretch{1.8}
    \begin{symbols}{*3{cl}}
        \hline
        \DEL{\lgroup}      & \DEL{\rgroup}      & \DEL{\lmoustache}  \\
        \DEL{\arrowvert}   & \DEL{\Arrowvert}   & \DEL{\bracevert} \\
        \DEL{\rmoustache} \\
        \hline
    \end{symbols}
\end{table}

\subsubsection{其他符号}
\begin{table}[H]
    \centering
    \caption{其他符号}\label{tbl:math-misc}
    \begin{symbols}{*4{cl}}
        \hline
        \SYM{\dots}       & \SYM{\cdots}      & \SYM{\vdots}      & \SYM{\ddots}     \\
        \SYM{\hbar}       & \SYM{\imath}      & \SYM{\jmath}      & \SYM{\ell}       \\
        \SYM{\Re}         & \SYM{\Im}         & \SYM{\aleph}      & \SYM{\wp}        \\
        \SYM{\forall}     & \SYM{\exists}     & \LSYM{\mho}       & \SYM{\partial}   \\
        \SYM{'}           & \SYM{\prime}      & \SYM{\emptyset}   & \SYM{\infty}     \\
        \SYM{\nabla}      & \SYM{\triangle}   & \LSYM{\Box}       & \LSYM{\Diamond}  \\
        \SYM{\bot}        & \SYM{\top}        & \SYM{\angle}      & \SYM{\surd}      \\
        \SYM{\diamondsuit} & \SYM{\heartsuit} & \SYM{\clubsuit}   & \SYM{\spadesuit} \\
        \SYM{\neg} or \verb|\lnot| & \SYM{\flat} & \SYM{\natural}    & \SYM{\sharp}     \\
        \hline
    \end{symbols}
\end{table}

\subsection{\hologo{AmS} 符号}

本小节所有符号依赖 \lstinline{amssymb} 宏包。

\subsubsection{希腊字母和希伯来字母}
\begin{table}[H]
    \centering
    \caption{\AmS{} 希腊字母和希伯来字母} \label{tbl:ams-greek-hebrew}
    \begin{symbols}{*5{cl}}
        \hline
        \AMSSYM{\digamma}   &\AMSSYM{\varkappa} & \AMSSYM{\beth} &\AMSSYM{\gimel} & \AMSSYM{\daleth}\\
        \hline
    \end{symbols}
\end{table}

\subsubsection{\AmS{} 二元关系符}
\begin{table}[H]
    \centering
    \caption{\AmS{} 二元关系符} \label{tbl:ams-rel}
    \begin{symbols}{*3{cl}}
        \hline
        \AMSSYM{\lessdot}           & \AMSSYM{\gtrdot}            & \AMSSYM{\doteqdot} \\
        \AMSSYM{\leqslant}          & \AMSSYM{\geqslant}          & \AMSSYM{\risingdotseq}     \\
        \AMSSYM{\eqslantless}       & \AMSSYM{\eqslantgtr}        & \AMSSYM{\fallingdotseq}    \\
        \AMSSYM{\leqq}              & \AMSSYM{\geqq}              & \AMSSYM{\eqcirc}           \\
        \AMSSYM{\lll} or \verb|\llless| & \AMSSYM{\ggg}               & \AMSSYM{\circeq}  \\
        \AMSSYM{\lesssim}           & \AMSSYM{\gtrsim}            & \AMSSYM{\triangleq}        \\
        \AMSSYM{\lessapprox}        & \AMSSYM{\gtrapprox}         & \AMSSYM{\bumpeq}           \\
        \AMSSYM{\lessgtr}           & \AMSSYM{\gtrless}           & \AMSSYM{\Bumpeq}           \\
        \AMSSYM{\lesseqgtr}         & \AMSSYM{\gtreqless}         & \AMSSYM{\thicksim}         \\
        \AMSSYM{\lesseqqgtr}        & \AMSSYM{\gtreqqless}        & \AMSSYM{\thickapprox}      \\
        \AMSSYM{\preccurlyeq}       & \AMSSYM{\succcurlyeq}       & \AMSSYM{\approxeq}         \\
        \AMSSYM{\curlyeqprec}       & \AMSSYM{\curlyeqsucc}       & \AMSSYM{\backsim}          \\
        \AMSSYM{\precsim}           & \AMSSYM{\succsim}           & \AMSSYM{\backsimeq}        \\
        \AMSSYM{\precapprox}        & \AMSSYM{\succapprox}        & \AMSSYM{\vDash}            \\
        \AMSSYM{\subseteqq}         & \AMSSYM{\supseteqq}         & \AMSSYM{\Vdash}            \\
        \AMSSYM{\shortparallel}     & \AMSSYM{\Supset}            & \AMSSYM{\Vvdash}           \\
        \AMSSYM{\blacktriangleleft} & \AMSSYM{\sqsupset}          & \AMSSYM{\backepsilon}      \\
        \AMSSYM{\vartriangleright}  & \AMSSYM{\because}           & \AMSSYM{\varpropto}        \\
        \AMSSYM{\blacktriangleright}& \AMSSYM{\Subset}            & \AMSSYM{\between}          \\
        \AMSSYM{\trianglerighteq}   & \AMSSYM{\smallfrown}        & \AMSSYM{\pitchfork}        \\
        \AMSSYM{\vartriangleleft}   & \AMSSYM{\shortmid}          & \AMSSYM{\smallsmile}       \\
        \AMSSYM{\trianglelefteq}    & \AMSSYM{\therefore}         & \AMSSYM{\sqsubset}         \\
        \hline
    \end{symbols}
\end{table}

\subsubsection{\hologo{AmS} 二元运算符}
\begin{table}[H]
    \centering
    \caption{\hologo{AmS} 二元运算符} \label{tbl:ams-op}
    \begin{symbols}{*3{cl}}
        \hline
        \AMSSYM{\dotplus}        & \AMSSYM{\centerdot}      &       \\
        \AMSSYM{\ltimes}         & \AMSSYM{\rtimes}         & \AMSSYM{\divideontimes} \\
        \AMSSYM{\doublecup}      & \AMSSYM{\doublecap}      & \AMSSYM{\smallsetminus} \\
        \AMSSYM{\veebar}         & \AMSSYM{\barwedge}       & \AMSSYM{\doublebarwedge}\\
        \AMSSYM{\boxplus}        & \AMSSYM{\boxminus}       & \AMSSYM{\circleddash}   \\
        \AMSSYM{\boxtimes}       & \AMSSYM{\boxdot}         & \AMSSYM{\circledcirc}   \\
        \AMSSYM{\intercal}       & \AMSSYM{\circledast}     & \AMSSYM{\rightthreetimes} \\
        \AMSSYM{\curlyvee}       & \AMSSYM{\curlywedge}     & \AMSSYM{\leftthreetimes} \\
        \hline
    \end{symbols}
\end{table}

\subsubsection{\hologo{AmS} 箭头}
\begin{table}[H]
    \centering
    \caption{\hologo{AmS} 箭头}\label{tbl:ams-arrows}
    \begin{symbols}{*2{cl}}
        \hline
        \AMSSYM{\dashleftarrow}      & \AMSSYM{\dashrightarrow}     \\
        \AMSSYM{\leftleftarrows}     & \AMSSYM{\rightrightarrows}   \\
        \AMSSYM{\leftrightarrows}    & \AMSSYM{\rightleftarrows}    \\
        \AMSSYM{\Lleftarrow}         & \AMSSYM{\Rrightarrow}        \\
        \AMSSYM{\twoheadleftarrow}   & \AMSSYM{\twoheadrightarrow}  \\
        \AMSSYM{\leftarrowtail}      & \AMSSYM{\rightarrowtail}     \\
        \AMSSYM{\leftrightharpoons}  & \AMSSYM{\rightleftharpoons}  \\
        \AMSSYM{\Lsh}                & \AMSSYM{\Rsh}                \\
        \AMSSYM{\looparrowleft}      & \AMSSYM{\looparrowright}     \\
        \AMSSYM{\curvearrowleft}     & \AMSSYM{\curvearrowright}    \\
        \AMSSYM{\circlearrowleft}    & \AMSSYM{\circlearrowright}   \\
        \AMSSYM{\multimap}           & \AMSSYM{\upuparrows}         \\
        \AMSSYM{\downdownarrows}     & \AMSSYM{\upharpoonleft}      \\
        \AMSSYM{\upharpoonright}     & \AMSSYM{\downharpoonright}   \\
        \AMSSYM{\rightsquigarrow}    & \AMSSYM{\leftrightsquigarrow}\\
        \hline
    \end{symbols}
\end{table}

\subsubsection{\hologo{AmS} 反义二元关系符和箭头}
\begin{table}[H]
    \centering
    \caption{\hologo{AmS} 反义二元关系符和箭头}\label{tbl:ams-negative}
    \begin{symbols}{*3{cl}}
        \hline
        \AMSSYM{\nless}           & \AMSSYM{\ngtr}            & \AMSSYM{\varsubsetneqq}    \\
        \AMSSYM{\lneq}            & \AMSSYM{\gneq}            & \AMSSYM{\varsupsetneqq}    \\
        \AMSSYM{\nleq}            & \AMSSYM{\ngeq}            & \AMSSYM{\nsubseteqq}       \\
        \AMSSYM{\nleqslant}       & \AMSSYM{\ngeqslant}       & \AMSSYM{\nsupseteqq}       \\
        \AMSSYM{\lneqq}           & \AMSSYM{\gneqq}           & \AMSSYM{\nmid}             \\
        \AMSSYM{\lvertneqq}       & \AMSSYM{\gvertneqq}       & \AMSSYM{\nparallel}        \\
        \AMSSYM{\nleqq}           & \AMSSYM{\ngeqq}           & \AMSSYM{\nshortmid}        \\
        \AMSSYM{\lnsim}           & \AMSSYM{\gnsim}           & \AMSSYM{\nshortparallel}   \\
        \AMSSYM{\lnapprox}        & \AMSSYM{\gnapprox}        & \AMSSYM{\nsim}             \\
        \AMSSYM{\nprec}           & \AMSSYM{\nsucc}           & \AMSSYM{\ncong}            \\
        \AMSSYM{\npreceq}         & \AMSSYM{\nsucceq}         & \AMSSYM{\nvdash}           \\
        \AMSSYM{\precneqq}        & \AMSSYM{\succneqq}        & \AMSSYM{\nvDash}           \\
        \AMSSYM{\precnsim}        & \AMSSYM{\succnsim}        & \AMSSYM{\nVdash}           \\
        \AMSSYM{\precnapprox}     & \AMSSYM{\succnapprox}     & \AMSSYM{\nVDash}           \\
        \AMSSYM{\subsetneq}       & \AMSSYM{\supsetneq}       & \AMSSYM{\ntriangleleft}    \\
        \AMSSYM{\varsubsetneq}    & \AMSSYM{\varsupsetneq}    & \AMSSYM{\ntriangleright}   \\
        \AMSSYM{\nsubseteq}       & \AMSSYM{\nsupseteq}       & \AMSSYM{\ntrianglelefteq}  \\
        \AMSSYM{\subsetneqq}      & \AMSSYM{\supsetneqq}      & \AMSSYM{\ntrianglerighteq} \\
        \AMSSYM{\nleftarrow}      & \AMSSYM{\nrightarrow}     & \AMSSYM{\nleftrightarrow}  \\
        \AMSSYM{\nLeftarrow}      & \AMSSYM{\nRightarrow}     & \AMSSYM{\nLeftrightarrow}  \\
        \hline
    \end{symbols}
\end{table}

\subsubsection{\hologo{AmS} 定界符}
\begin{table}[H]
    \centering
    \caption{\hologo{AmS} 定界符}\label{tbl:ams-delims}
    \begin{symbols}{*4{cl}}
        \hline
        \AMSSYM{\ulcorner} & \AMSSYM{\urcorner} & \AMSSYM{\llcorner} & \AMSSYM{\lrcorner} \\
        \hline
    \end{symbols}
\end{table}

\subsubsection{\hologo{AmS} 其它符号}
\begin{table}[H]
    \centering
    \caption{\hologo{AmS} 其它符号}\label{tbl:ams-misc}
    \begin{symbols}{*3{cl}}
        \hline
        \AMSSYM{\hbar}             & \AMSSYM{\hslash}           & \\%\AMSSYM{\Bbbk}            \\
        \AMSSYM{\square}           & \AMSSYM{\blacksquare}      & \AMSSYM{\circledS}        \\
        \AMSSYM{\vartriangle}      & \AMSSYM{\blacktriangle}    & \AMSSYM{\complement}      \\
        \AMSSYM{\triangledown}     & \AMSSYM{\blacktriangledown}& \AMSSYM{\Game}            \\
        \AMSSYM{\lozenge}          & \AMSSYM{\blacklozenge}     & \AMSSYM{\bigstar}         \\
        \AMSSYM{\angle}            & \AMSSYM{\measuredangle}    & \\
        \AMSSYM{\diagup}           & \AMSSYM{\diagdown}         & \AMSSYM{\backprime}       \\
        \AMSSYM{\nexists}          & \AMSSYM{\Finv}             & \AMSSYM{\varnothing}      \\
        \AMSSYM{\eth}              & \AMSSYM{\sphericalangle}   & \AMSSYM{\mho}             \\
        \hline
    \end{symbols}
\end{table}

\section{多行公式}\label{sec:multi-eqns}

目前最常用的是 \amsenv{align} 环境,它将公式用 \texttt\& 隔为两部分并对齐。分隔符通常放在等号左边:

\begin{codeshow}
    \begin{align}
        a & = b + c \\
          & = d + e
    \end{align}
\end{codeshow}

\amsenv{align} 环境会给每行公式都编号。我们仍然可以用 \amscmd{notag} 去掉某行的编号。
在以下的例子,为了对齐等号,我们将分隔符放在右侧,并且此时需要在等号后添加一对括号 \texttt\{\texttt\} 以产生正常的间距:

\begin{codeshow}
    \begin{align}
        a ={} & b + c                 \\
        ={}   & d + e + f + g + h + i
        + j + k + l \notag            \\
              & + m + n + o           \\
        ={}   & p + q + r + s
    \end{align}
\end{codeshow}

\amsenv{align} 还能够对齐多组公式,除等号前的 \texttt\& 之外,公式之间也用 \texttt\& 分隔:

\begin{codeshow}
    \begin{align}
        a & =1  & b & =2  & c & =3  \\
        d & =-1 & e & =-2 & f & =-5
    \end{align}
\end{codeshow}

如果我们不需要按等号对齐,只需罗列数个公式,\amsenv{gather} 将是一个很好用的环境:

\begin{codeshow}
    \begin{gather}
        a = b + c \\
        d = e + f + g \\
        h + i = j + k \notag \\
        l + m = n
    \end{gather}
\end{codeshow}

\amsenv{align} 和 \amsenv{gather} 有对应的不带编号的版本 \amsenv{align*} 和 \amsenv{gather*}。

\subsection{公用编号的多行公式}\label{subsec:aligned}

另一个常见的需求是将多个公式组在一起公用一个编号,编号位于公式的居中位置。为此,\lstinline{amsmath} 宏包
提供了诸如 \amsenv{aligned}、\amsenv{gathered} 等环境,与 \lstinline{equation} 环境套用。
以 \texttt{-ed} 结尾的环境用法与前一节不以 \texttt{-ed} 结尾的环境用法一一对应。我们仅以 \amsenv{aligned} 举例:

\begin{codeshow}
    \begin{equation}
        \begin{aligned}
            a     & = b + c     \\
            d     & = e + f + g \\
            h + i & = j + k     \\
            l + m & = n
        \end{aligned}
    \end{equation}
\end{codeshow}

\amsenv{split} 环境和 \amsenv{aligned} 环境用法类似,也用于和 \lstinline{equation} 环境套用,
区别是 \amsenv{split} 只能将每行的一个公式分两栏,\amsenv{aligned} 允许每行多个公式多栏。









\section{}