% !TEX root = ../latexnote.tex
\chapter{版面和格式}\label{chap:format}



\section{字体设置}\label{sec:font}
\subsection{使用中文}\label{subsec:chinese}
使用中文需要导入ctex宏包。
\begin{lstlisting}
\usepackage{ctex}
\end{lstlisting}

\subsection{全局字体设置}\label{subsec:global}

\begin{lstlisting}
  \setmainfont{Times New Roman}	%仅对西文起作用,需要fontspec宏包
  \setCJKmainfont{SimSun}			%仅对中文起作用
\end{lstlisting}

\subsection{局部字体设置}\label{subsec:local}

\begin{lstlisting}
  \setCJKfamilyfont{zhsong}{FZShuSong-Z01}
  \newcommand*{\songti}{\CJKfamily{zhsong}}
\end{lstlisting}

可以使用\lstinline{\songti}命令来调用宋体字体。

\subsection{文字装饰}\label{subsec:decor}

\LaTeX{} 提供\lstinline{\underline}命令设置下划线。

\begin{codeshow}
  \underline{underlined text}
\end{codeshow}

\lstinline{\underline}命令提供的下划线样式不够灵活,ulem或CJKfntef宏包提供了更灵活的下划线命令,CJKfntef宏包对中文支持更好。

\begin{codeshow}
  \CJKunderdot{important 非常重要}\\
  \CJKunderline{notice 注意}\\
  \CJKunderdblline{urgent 必须}\\
  \CJKunderwave{prompt 提示}\\
  \CJKsout{wrong 错误}\\
  \CJKxout{removed 删除}
\end{codeshow}

\section{分栏}\label{sec:columns}

\LaTeX{} 支持简单的单栏或双栏排版。标准文档类的全局选项 onecolumn、twocolumn 可控制全文分单栏或双栏排版。LATEX 也提供了切换单/双栏排版的命令:

\begin{lstlisting}
\onecolumn
\twocolumn
\end{lstlisting}

切换单/双栏排版时总是会另起一页(\lstinline{\clearpage})。在双栏模式下使用 \lstinline{\newpage} 会换栏而不是换页;\lstinline{\clearpage} 则能够换页。

使用multicol宏包可以实现多栏排版。并且切换不会另起一页。

\begin{lstlisting}
\usepackage{multicol}
\begin{multicols}{2}
  ...
\end{multicols}
\end{lstlisting}

\section{断行、断页、断词}\label{sec:break}

\begin{lstlisting}
\\[length] % \\可以带可选参数length,用于在断行处添加垂直间距,可以在表格公式等地方使用
\newline %不带可选参数,只能用于文本段落中

\newpage %双栏模式下另起一栏,单栏模式下另起一页
\clearpage %单栏、双栏模式下都另起一页
\end{lstlisting}

如果 \LaTeX{} 遇到了很长的英文单词,仅在单词之间的“空格”处断行无法生成疏密程度匀称的段落时,就会考虑从单词中间断开。对于绝大多数单词,\LaTeX{} 能够找到合适的断词位置,在断开的行尾加上连字符\lstinline{-}。如果一些单词没能自动断词,我们可以在单词内手动使用 \lstinline{\-} 命令指定断词的位置:

\begin{codeshow}
  I think this is: su\-per\-cal\-%
  i\-frag\-i\-lis\-tic\-ex\-pi\-%
  al\-i\-do\-ciousu\-per\-cal\-%
  i\-frag\-i\-lis\-tic\.
\end{codeshow}

\section{空格和分段}\label{sec:space}
\LaTeX{} 源代码中,空格键和 Tab 键输入的空白字符视为“空格”。连续的若干个空白字符视为一个空格。一行开头的空格忽略不计。

行末的换行符视为一个空格;但连续两个换行符,也就是空行,会将文字分段。多个空行被视为一个空行。也可以在行末使用 \lstinline{\par} 命令分段。

\begin{codeshow}
  This is a new paragraph. Text text text
  text text   text text text text text text.

  This is a new paragraph.\par
  This is a new paragraph.
\end{codeshow}

\section{注释}\label{sec:comment}
LATEX 用 \% 字符作为注释。在这个字符之后直到行末,所有的字符都被忽略,行末的换行符也不引入空格。

\begin{codeshow}
  This is an % short comment
  % ---
  % Long and organized
  % comments
  % ---
  example: Comments do not bre%
  ak a word.
\end{codeshow}

\section{特殊字符}\label{sec:special}

以下字符在 LATEX 里有特殊用途,如 \% 表示注释,\$、\^{}、\_ 等用于排版数学公式,\& 用于排版表格,等等。直接输入这些字符得不到对应的符号,还往往会出错:

\begin{lstlisting}
  # $ % & { } _ ^ ~ \
\end{lstlisting}

如果想要输入以上符号,需要使用以下带反斜线的形式输入,类似编程语言里的“转义”符号:
\begin{lstlisting}
  \# \$ \% \& \{ \} \_
  \^{} \~{} \textbackslash
\end{lstlisting}

\section{问题}\label{sec:qa}
\subsection{双面空白页}\label{subsec:blankpage}
\qaq{问题:}采用\lstinline{twoside}文档类,使用\lstinline{\cleardoublepage}设置双面空白页,如何去除空白页?

\qaq{解决方法:}在所有\lstinline{\cleardoublepage}之前添加\lstinline{\let\cleardoublepage\clearpage}。